\section{第三方模块}
\texttt{hi-nginx}目前集成了一些常用的第三方模块。包括:
\begin{itemize}
\item array-var-nginx-module
\item form-input-nginx-module
\item headers-more-nginx-module
\item iconv-nginx-module
\item memc-nginx-module
\item ngx_coolkit
\item ngx_devel_kit
\item rds-csv-nginx-module
\item rds-json-nginx-module
\item redis2-nginx-module
\item set-misc-nginx-module
\item srcache-nginx-module
\item xss-nginx-module
\item nginx-http-concat
\item nginx-push-stream-module
\item nginx-rtmp-module
\item nchan
\item echo-nginx-module
\item nginx-upload-module
\end{itemize}
它们都放在\path{3rd}目录中,需要时,只需\texttt{--add-module=3rd/xxx}即可。例如:
\begin{lstlisting}
#!/bin/bash
./configure     --with-http_ssl_module \
                --with-http_v2_module \
                --prefix=/home/centos7/nginx \
                --add-module=ngx_http_hi_module  \
                --add-module=3rd/ngx_devel_kit-0.3.0 \
                --add-module=3rd/nginx-http-concat-1.2.2 \
                --add-module=3rd/echo-nginx-module-0.60 \
                --add-module=3rd/array-var-nginx-module-0.05 \
                --add-module=3rd/form-input-nginx-module-0.12  \
                --add-module=3rd/headers-more-nginx-module-0.32 \
                --add-module=3rd/iconv-nginx-module-0.14 \
                --add-module=3rd/memc-nginx-module-0.18 \
                --add-module=3rd/ngx_coolkit-0.2rc3 \
                --add-module=3rd/rds-csv-nginx-module-0.07 \
                --add-module=3rd/rds-json-nginx-module-0.14 \
                --add-module=3rd/redis2-nginx-module-0.14 \
                --add-module=3rd/set-misc-nginx-module-0.31 \
                --add-module=3rd/srcache-nginx-module-0.31 \
                --add-module=3rd/xss-nginx-module-0.05 \
                --add-module=3rd/nginx-push-stream-module-0.5.2 \
                --add-module=3rd/nchan-1.1.6 \
                --add-module=3rd/nginx-rtmp-module-1.1.7.10 \
                --add-module=3rd/nginx-upload-module-2.255
\end{lstlisting}
如果用户有需要集成的其他模块,最好也放在\path{3rd}目录中,便于统一管理。然后,接着上面的配置脚本,\texttt{--add-module=3rd/xxx},依样画葫芦即可。

\section{模块开发}
一般而言,\texttt{nginx}模块开发以\texttt{c}语言为基础。但是,对\texttt{hi-nginx}而言,\texttt{c}和\texttt{c++}都是完全支持的。\texttt{hi-nginx}本身即以\texttt{c++}模块\texttt{ngx_http_hi_module}为基础,它可以作为用户使用\texttt{c++}为\texttt{hi-nginx}开发自定义模块的范例演示。

实际上,把\texttt{ngx_http_hi_module}放在\path{3rd}目录中也是完全可以的。

\section{演示代码}
\texttt{hi-nginx}配有较为完整的演示代码。请参考\url{https://github.com/webcpp/hi_demo}



