\section{起步}
\subsection{hi-project}
\texttt{hi-project}是一个辅助脚本,安装在\path	{/home/centos7/nginx/hi}目录中。它的用途是为用户创建“起步”代码模板。运行它可以使用三个选项,依次是:
\begin{itemize}
\item 工程名,可选,默认\texttt{demo}
\item 工程类型,可选,支持\texttt{cpp},\texttt{python},\texttt{lua}和\texttt{java},默认\texttt{cpp}
\item \texttt{hi-nginx}安装路径,可选,默认\path{/home/centos7/nginx}
\end{itemize}
用户可以通过\texttt{-h}或者\texttt{--help}参看使用说明。

为了利用\texttt{hi-project}提供的便利,最好是在一个新建的空目录中运行它——它会在运行目录下创建一个总的Makefile,这个文件能够控制所有的子项目。

\subsection{hello world}
\texttt{hello world}工程包含了\texttt{hi-nginx} \texttt{web application}开发的最基本要素。
\subsubsection{cpp}
使用\texttt{hi-project}脚本创建一个\texttt{cpp}工程,名为\texttt{hello}:
\begin{lstlisting}
/home/centos7/nginx/hi/hi-project hello cpp /home/centos7/nginx
\end{lstlisting}
后面两个参数是可省的。
这时,\texttt{hi-project}会创建一个名为\path{hello}的目录,并在该目录中创建两个文件,一个是\path{Makefile},一个是\path{hello.cpp}。前者帮助用户在执行\texttt{make \&\& make install }时把\texttt{web application}编译、安装至正确位置;后者则帮助用户正确创建合乎\texttt{hi-nginx}要求的\texttt{class}:
\begin{lstlisting}
#include "servlet.hpp"


namespace hi {

    class hello : public servlet {
    public:

        void handler(request& req, response& res) {
            res.headers.find("Content-Type")->second = "text/plain;charset=UTF-8";
            res.content = "hello,world";
            res.status = 200;
        }

    };
}

extern "C" hi::servlet* create() {
    return new hi::hello();
}

extern "C" void destroy(hi::servlet* p) {
    delete p;
}
\end{lstlisting}
以上代码一目了然,无需过多解释,任何熟知\texttt{c++}的程序员都能看懂。没错,\texttt{hi-nginx}并不要求\texttt{cpp}程序员“精通”自己的工具,只需熟知即可。当然,熟知\texttt{http}协议是必要的,否则很难正确地使用\texttt{request}类和\texttt{response}类。如何使用这两个类,下文会详述。在此先按下不表。

用户\texttt{make \&\& make install}之后,\texttt{hi-project}后把编译生成的\path{hello.so}安装在\path{/home/centos7/nginx/hi}目录下。要启用这个\texttt{web application},只需在\texttt{hi-nginx}的配置文件中加入:
\begin{lstlisting}
location = /hello {
	hi hi/hello.so;
}
\end{lstlisting}
然后,\texttt{hi-nginx}执行\texttt{reload} 即可。


\subsubsection{python}
使用\texttt{hi-project}脚本创建一个\texttt{python}工程,名为\texttt{python}:
\begin{lstlisting}
/home/centos7/nginx/hi/hi-project python python /home/centos7/nginx
\end{lstlisting}
最后一个参数是可省的。\texttt{hi-project}会创建一个名为\path{python}的目录,并在该目录中创建两个文件,一个是\path{Makefile},一个是\path{python.py}。前者的作用如上。后者很简单:
\begin{lstlisting}
hi_res.status(200)
hi_res.content('hello,python')
\end{lstlisting}
用户可以把其他\texttt{python}类型的\texttt{web}应用全放在\path{python}目录下,它们都能被正确安装。

\subsubsection{lua}
使用\texttt{hi-project}脚本创建一个\texttt{lua}工程,名为\texttt{lua}:
\begin{lstlisting}
/home/centos7/nginx/hi/hi-project luahello lua /home/centos7/nginx
\end{lstlisting}
最后一个参数是可省的。\texttt{hi-project}会创建一个名为\path{lua}的目录,并在该目录中创建两个文件,一个是\path{Makefile},一个是\path{lua.lua}。前者的作用如上。后者很简单:
\begin{lstlisting}
hi_res:status(200)
hi_res:content('hello,lua')
\end{lstlisting}
用户可以把其他\texttt{lua}类型的\texttt{web}应用全放在\path{lua}目录下,它们都能被正确安装。

\subsubsection{java}
使用\texttt{hi-project}脚本创建一个\texttt{jhello}工程,名为\texttt{jhello}:
\begin{lstlisting}
/home/centos7/nginx/hi/hi-project jhello java /home/centos7/nginx
\end{lstlisting}
最后一个参数是可省的。\texttt{hi-project}会创建一个名为\path{jhello}的目录,并在该目录中创建一个是\path{Makefile}。还会创建一个名为\path{hi}的子目录,该目录下会有一个名为\path{jhello.java}的源文件。前者的作用如上。后者很简单:
\begin{lstlisting}
package hi;

public class jhello implements hi.servlet {

    public jhello() {

    }

    public void handler(hi.request req, hi.response res) {
        res.status = 200;
        res.content ="hello,world"

    }
}
\end{lstlisting}
用户可以把其他\texttt{java}类型的\texttt{web}应用全放在\path{hi}目录下,它们都能被正确安装。

\subsection{请求类}
\texttt{hi-nginx}把请求分解为以下几个部分:
\begin{itemize}
\item 一般信息
	\begin{itemize}
		\item client
		\item user_agent
		\item method
		\item uri
		\item param
	\end{itemize}
\item headers头信息,需要开启\texttt{hi_need_headers}
\item form表单信息
\item cookies信息,需要开启\texttt{hi_need_cookies}
\item session会话信息,需要开启\texttt{hi_need_session}并配置\texttt{hi_redis_host}和\texttt{hi_redis_port}
\end{itemize}
用户可以通过配合“开关指令”获取想要的信息。
\subsubsection{python api}
\texttt{python}用户可以通过类实例\texttt{hi_req}结合以下方法获取想要的信息:
\begin{itemize}
\item client
\item user_agent
\item method
\item uri
\item param
\item has_xxx 
\item get_xxx
\end{itemize}
其中\texttt{xxx}可取值\texttt{header},\texttt{form},\texttt{session}和\texttt{cookie}。
\subsubsection{lua api}
\texttt{lua}用户可以通过类实例\texttt{hi_req}结合以下方法获取想要的信息:
\begin{itemize}
\item client
\item user_agent
\item method
\item uri
\item param
\item has_xxx 
\item get_xxx
\end{itemize}
其中\texttt{xxx}可取值\texttt{header},\texttt{form},\texttt{session}和\texttt{cookie}。
\subsection{响应类}
\texttt{hi-nginx}把响应分解为四个部分:
\begin{itemize}
\item status,状态码
\item headers,响应头
\item content,响应体
\item session,会话信息
\end{itemize}
用户通过以上四个变量操纵\texttt{hi-nginx}应答请求。
\subsubsection{python api}
\texttt{python}用户可以通过类实例\texttt{hi_res}结合以下方法写入想要的响应:
\begin{itemize}
\item status
\item header
\item content
\item session
\end{itemize}
\subsubsection{lua api}
\texttt{lua}用户可以通过类实例\texttt{hi_res}结合以下方法写入想要的响应:
\begin{itemize}
\item status
\item header
\item content
\item session
\end{itemize}


\subsection{指令详解}
\subsubsection{hi}
这个指令负责装载\texttt{cpp application}。它接受一个参数,参数值为动态\texttt{*.so}的动态链接库。
\subsubsection{hi_need_cache}
这个指令负责开关缓存管理器。\texttt{hi-nginx}的缓存管理器采用的时间过期和最近最少使用算法相结合的缓存管理策略。用户可以结合以下命令控制缓存管理器
\begin{itemize}
\item \texttt{hi_cache_size}
\item \texttt{hi_cache_expires}
\end{itemize}
\subsubsection{hi_need_headers}
这个指令负责开关http头信息获取。
\subsubsection{hi_need_cookies}
这个指令负责开关http cookie信息获取
\subsubsection{hi_need_session}
这个指令负责开关http session会话管理器。\texttt{hi-nginx}通过\texttt{redis}服务器来管理会话数据,因此,用户应该在打开这个开关的同时配置以下两项:
\begin{itemize}
\item \texttt{hi_redis_host}
\item \texttt{hi_redis_port}
\end{itemize}
此外,用户还可以通过\texttt{hi_session_expires}指令配置会话过期时间。

会话管理器与\texttt{hi_need_cookies}是同步打开的,\texttt{hi-nginx}需要通过cookie设置客户端唯一识别ID,而且限定识别ID必须具有SESSIONID名。用户可以使用\texttt{nginx}内置模块\texttt{ngx_http_userid_module}来配置识别ID,例如:
\begin{lstlisting}
	userid on;
	userid_name SESSIONID;
	userid_domain localhost;
	userid_path /;
\end{lstlisting}
\subsubsection{hi_py_content}
这个指令负责运行\texttt{python}内容块
\subsubsection{hi_py_script}
这个指令负责运行\texttt{python}脚本,它接受一个参数,可指定查找脚本的目录。
\subsubsection{hi_lua_content}
这个指令负责运行\texttt{lua}内容块
\subsubsection{hi_lua_script}
这个指令负责运行\texttt{lua}脚本,它接受一个参数,可指定查找脚本的目录。
\subsubsection{hi_java_classpath}
这个指令负责为虚拟机设置\texttt{CLASSPATH},一般的写法是:
\begin{lstlisting}
hi_java_classpath "-Djava.class.path=.:/home/centos7/nginx/java:/home/centos7/nginx/java/hi-nginx-java.jar";
\end{lstlisting}
\path{hi-nginx-java.jar}是必须添加的。其他jar文件可以自行添加。
\subsubsection{hi_java_options}
这个指令可指定虚拟机初始化参数,默认值为

\begin{lstlisting}
-server -d64 -Xmx3G -Xms3G -Xmn768m -XX:+DisableExplicitGC -XX:+UseConcMarkSweepGC -XX:+UseParNewGC -XX:+UseNUMA -XX:+CMSParallelRemarkEnabled -XX:MaxTenuringThreshold=15 -XX:MaxGCPauseMillis=30 -XX:GCPauseIntervalMillis=150 -XX:+UseAdaptiveGCBoundary -XX:-UseGCOverheadLimit -XX:+UseBiasedLocking -XX:SurvivorRatio=8 -XX:TargetSurvivorRatio=90 -XX:MaxTenuringThreshold=15 -Dfml.ignorePatchDiscrepancies=true -Dfml.ignoreInvalidMinecraftCertificates=true -XX:+UseFastAccessorMethods -XX:+UseCompressedOops -XX:+OptimizeStringConcat -XX:+AggressiveOpts -XX:ReservedCodeCacheSize=2048m -XX:+UseCodeCacheFlushing -XX:SoftRefLRUPolicyMSPerMB=10000 -XX:ParallelGCThreads=10
\end{lstlisting}

\subsubsection{hi_java_servlet_cache_expires}
这个指令负责安排\texttt{servlet}缓存时间,默认300秒,使用它可以周期性的自动更新\texttt{servlet class}。如果你修改了\texttt{servlet},更新安装后可以实现自动更新,而无需\texttt{reload}。
\subsubsection{hi_java_servlet_cache_size}
这个指令可指定\texttt{servlet}缓存容器大小。
\subsubsection{hi_java_version}
这个指令可指定虚拟机版本,默认是8,即\texttt{java}8。可以指定6,即\texttt{java}6。
\subsubsection{hi_java_servlet}
这个指令负责调用\texttt{servlet}类。比如,
\begin{lstlisting}
hi_java_servlet  hi/jhello;
\end{lstlisting}
它表示调用的是\texttt{hi-nginx}安装目录下的\path{java/hi/jhello.class}。














